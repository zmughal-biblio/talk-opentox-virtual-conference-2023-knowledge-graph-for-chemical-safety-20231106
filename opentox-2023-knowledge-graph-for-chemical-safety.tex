%% vim:tw=66:spell:wrap:ft=tex:
\ifx \printpresenthandout \undefined
	\ifx \printpresentarticle \undefined
		% no handout and no article
		\documentclass{beamer}
	\else
		% print article
		\documentclass[11pt]{article}
		\usepackage{beamerarticle}
	\fi
\else
	% handout
	\documentclass[handout]{beamer}
\fi
% Preamble
%\usepackage[small,sf,bf]{titlesec}
\PassOptionsToPackage{table,x11names,dvipsnames,rgb}{xcolor}

\usepackage[utf8]{inputenc}
\usepackage[british]{babel}

\usepackage{xstring}

\usepackage{tikz}
\usetikzlibrary{arrows,shapes,automata,positioning,trees,shadows,mindmap,decorations,calc}
\usepackage{colortbl}

\usepackage{float}
\usepackage{graphicx}
\usepackage[normalem]{ulem}

%% use as
%%     \vertcenterimage{\includegraphics{*}}
\newcommand{\vertcenterimage}[1]{\raisebox{-.5\height}{#1}}

%% use as
%%     \flipbox{\includegraphics{*}}
\newcommand{\flipbox}[1]{\scalebox{1}[-1]{#1}}

\usepackage{amsmath}
\usepackage{amsfonts}
\usepackage{bm} % bold mathematics (\bm command)

% set-builder notation using \Set{ ... | ... }
\usepackage{braket}

\usepackage{mathtools}
\usepackage{breqn}

%% make everything in the box bold; both the text and mathematics: using
%% \boldmath from `bm` package
\newcommand{\be}{\bfseries\boldmath}

\newcommand\computertext[1]{\texttt{#1}}
\newcommand\computertextfamily{\ttfamily}

\usepackage{diagbox} % \backslashbox in tables

\usepackage{pifont}% http://ctan.org/pkg/pifont
\newcommand{\cmark}{\ding{51}}%
\newcommand{\xmark}{\ding{55}}%

\newcommand{\todo}[1]{%
	\textcolor{red}{TODO: #1}%
}
\newcommand{\todofig}[1]{%
	\textcolor{red}{TODO figure: \nolinkurl{#1}}%
}

\usepackage[inline]{enumitem}


\usepackage{hyperref}
\hypersetup{%
	pdfauthor={Zakariyya Mughal},%
	pdfpagemode={UseNone},%
	pdfpagelayout={SinglePage}%
}


\usetheme{Ilmenau}
\usecolortheme{beaver}
\usepackage{ifxetex}

\ifxetex
	% set font to Tahoma
	\usefonttheme{professionalfonts} % using non standard fonts for beamer
	\usefonttheme{serif} % default family is serif
	\usepackage{fontspec}
\else
	\usepackage[T1]{fontenc}
\fi

% Title on title slide
\setbeamerfont{title}{size = \Large}
\setbeamercolor{title}{fg = black, bg = white}

\setbeamertemplate{headline}{}
\beamertemplatenavigationsymbolsempty

%\usefonttheme[onlymath]{serif}
\ifx \printpresentarticle \undefined
	\setbeamertemplate{frametitle}[default][center]
	\setbeamertemplate{footline}{}
	%\setbeamertemplate{footline}[frame number]

	%% Change beamer bullets to circles rather than the ball default
	\setbeamertemplate{itemize items}[circle]
	\setbeamertemplate{enumerate items}[circle]
\fi


\usepackage{textcomp}
\usepackage{fancyvrb}
\usepackage{changepage}
\usepackage{multicol}
%\usepackage{wasysym}
%\usepackage{listings}

%\lstset{%basicstyle=\small\ttfamily,
%%numbers=left,
%%escapeinside=||
%}
\newenvironment{indented}{\begin{adjustwidth}{1.5em}{}}{\end{adjustwidth}}

% http://tex.stackexchange.com/questions/12550/changing-default-width-of-blocks-in-beamer/12551#12551
\newenvironment<>{varblock}[2][.9\textwidth]{%
  \setlength{\textwidth}{#1}
  \begin{actionenv}#3%
    \def\insertblocktitle{#2}%
    \par%
    \usebeamertemplate{block begin}}
  {\par%
    \usebeamertemplate{block end}%
  \end{actionenv}}

\ifx \printpresentnote \undefined
% no notes
\else
\setbeameroption{show only notes}
\fi

%% use as
%%     \af{ number of overlay }{ slide label }
%% e.g.,
%%     \af{2}{intro-slide}
\newcommand{\af}[2]{\againframe<#1|handout:0>[noframenumbering]{#2}}

%% TikZ arrows
%% From <https://tex.stackexchange.com/questions/61507/drawing-arrows-in-beamer>
\tikzset{
    myarrow/.style={
        draw,
        fill=orange,
        single arrow,
        minimum height=3.5ex,
        single arrow head extend=1ex
    }
}
\newcommand{\arrowup}{%
\tikz [baseline=-0.5ex]{\node [myarrow,rotate=90] {};}
}
\newcommand{\arrowdown}{%
\tikz [baseline=-1ex]{\node [myarrow,rotate=-90] {};}
}
\newcommand{\arrowright}{%
\tikz [baseline=-0.5ex]{\node [myarrow,rotate=0] {};}
}
\newcommand{\arrowleft}{%
\tikz [baseline=-0.5ex]{\node [myarrow,rotate=180] {};}
}


%% no "Figure" in \caption{}
%% simple caption
\setbeamertemplate{caption}{\raggedright\insertcaption\par}
%% with color
%\setbeamertemplate{caption}{%
%\begin{beamercolorbox}[wd=.5\paperwidth, sep=.2ex]{block
%body}\insertcaption%
%\end{beamercolorbox}%
%}

%\usepackage{listings}
%\lstset{language=Shell,%
%basicstyle=\footnotesize\singlespacing%
%}

\usepackage[toc,acronym]{glossaries}

\usepackage{qrcode}

% Beamer hack : <https://tex.stackexchange.com/questions/3455/problem-with-enumerate-package-in-beamer-class>
\def\labelenumi{\theenumi}

\begin{document}
% {{{ Meta
% meta needs to be in \begin{document} so that tabular works
\title[BioBricks-OKG]{BioBricks-OKG: An Open Knowledge Graph for Cheminformatics and Chemical Safety}
\author[Zaki Mughal]{Zakariyya Mughal}
\institute[Insilica]{Insilica}
\date[2023 Nov 06]{2023 Nov 06 \\[1ex]
\href{https://opentox.net/events/virtual-conference-2023/program}{OpenTox Virtual Conference 2023} \\[2ex]
}
%\qrcode{https://github.com/zmughal-biblio/talk-opentox-virtual-conference-2023-knowledge-graph-for-chemical-safety-20231106}\\[1ex]
%\btVFill
%
%{\footnotesize\url{https://github.com/zmughal-biblio/talk-opentox-virtual-conference-2023-knowledge-graph-for-chemical-safety-20231106}}
%}
% OpenTox Virtual Conference 2023 (6 November 2023 - 10 November 2023)

%% transitions are made transparent rather than hidden
%\setbeamercovered{transparent}
%  - Title slide.%{{{
\frame{\titlepage}
%}}}

\section{Problem space}
\begin{frame}\frametitle{\secname}
	Before you can answer a particular cheminformatics
	question:
	\begin{description}
		\item[P0] Many different data sources for
			cheminformatics and chemical safety.
		\pause
		\item[P1] Retrieving data sources for local
			analysis.
		\pause
		\item[P2] Differences in how data are represented.
	\end{description}
\end{frame}

\section{Solution: tabular data}
\begin{frame}\frametitle{\secname}
	\begin{itemize}
		\item multiple data sources
		\item local analysis
	\end{itemize}
	\pause
	\begin{center}
		BioBricks
	\end{center}
\end{frame}

\section{Solution: graph data}
\begin{frame}\frametitle{\secname}
	\begin{itemize}
		\item multiple data sources
		\item local analysis
		\pause
		\item \textbf{interoperable data through common
			vocabularies, identifiers, and ontologies}
	\end{itemize}
	\pause
	\begin{center}
		BioBricks-OKG
	\end{center}
\end{frame}

\section{BioBricks-OKG}
\begin{frame}{\secname}
	\begin{itemize}
		\item Working as part of NSF Proto-OKN program
		\pause
		\item Agency partner: NICEATM
		\pause
		\item Built on RDF metadata standard
	\end{itemize}
\end{frame}

\begin{frame}[fragile]{RDF}
  \begin{itemize}
  \item RDF based on triple relationships
  \end{itemize}
  \begin{figure}[tbp]
   \centering
   \resizebox{0.4\textwidth}{!}{\usetikzlibrary{shapes,positioning,arrows,calc}
\begin{tikzpicture}[>= triangle 60,
  remember picture,
  every node/.style={scale=0.7},
  ]
	\node [draw,circle] (subject) {subject} ;
	\node [draw,rectangle,right=of subject] (predicate) {predicate} ;
	\node [draw,circle,right=of predicate] (object) {object} ;
	\draw [->] (subject) -- (predicate) -- (object) ;
\end{tikzpicture}
}
   %\caption{\textbf{}}
  \label{fig:rdf}
  \end{figure}
\end{frame}

\section{Technology}
\begin{frame}{\secname}
  \begin{enumerate}
  \def\labelenumi{\arabic{enumi}.}
  \item Build on existing BioBricks infrastructure
  \item Use tabular data as a data source (tabular brick)
  \item Map tabular data graph data (triple brick)
  \item Load this triple brick into a triplestore
  graph database or use directly
  \end{enumerate}
\end{frame}

\section{Knowledge organization}
\begin{frame}[fragile]{\secname}
        \begin{figure}[tbp]%
        \centering%
        \usetikzlibrary{shapes,positioning,arrows,calc}
\begin{tikzpicture}[>= triangle 60,
  remember picture,
  every node/.style={scale=0.65},
  mygroup/.style={ draw,fill=yellow!20 }
  ]
	\node [mygroup] (biology) {
		\begin{tikzpicture}
			\node at (0,2) {Biology} ;
			\node [draw,fill=gray!30, rectangle split, rectangle split parts=6,anchor=center]
				(bio-types)
			{
				\nodepart{one}   Species
				\nodepart{two}   Anatomy
				\nodepart{three} Genes
				\nodepart{four}  Proteins
				\nodepart{five}  Pathways
				\nodepart{six}   Phenotypes
			} ;
			\node [right=of bio-types, fill=green!30, draw,rounded corners,rectangle,align=center]
			{
				%\begin{center}
				\begin{minipage}[c]{0.40\textwidth}
					\begin{itemize}[label={}]
						\item MeSH
						\item Uberon
						\item NCI Thesaurus
						\item NCBI Taxonomy
						\item HGNC
						\item Protein Ontology
					\end{itemize}
				\end{minipage}
				%\end{center}
			} ;
		\end{tikzpicture}
	};
	\node [mygroup,below=2 cm of biology]
		(chemistry) {
		\begin{tikzpicture}
			\node at (0,2) {Chemistry} ;
			\node [draw,fill=gray!30, rectangle split, rectangle split parts=4,anchor=center]
				(chem-types)
			{
				\nodepart{one}   Chemical Identifier
				\nodepart{two}   Quantities
				\nodepart{three} Structure
				\nodepart{four}  Reactions
			} ;
			\node [right=of chem-types, fill=green!30, draw,rounded corners,rectangle,align=center]
			{
				%\begin{center}
				\begin{minipage}[c]{0.40\textwidth}
					\begin{itemize}[label={}]
						\item PubChem
						\item CAS
						\item InChIKey
						\item MeSH
						\item QUDT
						\item Cheminformatics
							Ontology\\(CHEMINF)
						\item SMILES
					\end{itemize}
				\end{minipage}
				%\end{center}
			} ;
		\end{tikzpicture}
	} ;
	\node [mygroup,below right=0.3 cm and 0.8 cm of biology]
		(toxicology) {
		\begin{tikzpicture}
			\node at (0,2) {Toxicology} ;
			\node [draw,fill=gray!30, rectangle split, rectangle split parts=6,anchor=center]
				(tox-types)
			{
				\nodepart{one}   Evidence
				\nodepart{two}   Methods (assay)
				\nodepart{three} Exposure (e.g., route)
				\nodepart{four}  Chemical-Gene Interactions
				\nodepart{five}  Testing Outcomes
				\nodepart{six}   Toxicokinetics/dynamics
			} ;
			\node [right=of tox-types, fill=green!30, draw,rounded corners,rectangle,align=center]
			{
				\begin{minipage}[c]{0.50\textwidth}
					\begin{itemize}[label={}]
						\item PubMed
						\item DSSTox Chemical Identifier
						\item Evidence and Conclusion (ECO)
						\item Environment (ENVO)
						\item Exposure (ExO)
						\item Environmental conditions, treatments and exposures (ECTO)
						\item httk
					\end{itemize}
				\end{minipage}
			} ;
		\end{tikzpicture}
	}; 
	\draw [->] (biology)  edge (toxicology) ;
	\draw [->] (chemistry) edge (toxicology) ;
\end{tikzpicture}
%
        %%\caption{\textbf{Knowledge organization}}\label{fig:knowledge-org}%
        \end{figure}
\end{frame}

\section{Early prototype}
\begin{frame}{\secname}
	\begin{itemize}
		\def\labelenumi{\arabic{enumi}.}
		\item datasets: ICE, tox21, CTDbase, MeSH, HGNC, CosIng
		\pause
		\item Looking for
		\begin{itemize}
			\item New datasets
			\item New use cases
			\item Improvements to data modeling
		\end{itemize}
	\end{itemize}
\end{frame}


%%% }}}
%%% {{{ END

% Blank frame
%\begin{frame}\frametitle{\secname}
%\end{frame}

\end{document}
